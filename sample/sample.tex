\documentclass[dvipdfmx,11pt]{beamer}
\usepackage{bxdpx-beamer}% dvipdfmxなので必要
\usepackage{pxjahyper}% 日本語で'しおり'したい
\usepackage{amsmath, amssymb, amsthm}

\renewcommand{\kanjifamilydefault}{\gtdefault}% 既定をゴシック体に

\usetheme{TsukubaMin}

\theoremstyle{plain}
\newtheorem{theo}{定理}
\newtheorem{propo}{命題}
\newtheorem{lemm}{補題}

%タイトル
\title{サンプル}
\subtitle{sample}
\author{閔 正媛}
\date{2017/11/13}
\begin{document}
\begin{frame}
	\titlepage
\end{frame}

\begin{frame}{自己紹介}
\begin{itemize}
\item 2007年 筑波大学理工学群数学科入学
\item 2011年 筑波大学卒業
\item 大学1年のころ「ケーナ」という南米の管楽器に出会う
\end{itemize}
\end{frame}

\begin{frame}{ダミー定理ページ}
\begin{theo}
dummy theorem ダミー定理
\end{theo}
\begin{proof}
ダミー証明
\end{proof}
\end{frame}

\end{document}

